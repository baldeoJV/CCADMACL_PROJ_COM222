\documentclass{article}
\usepackage{amsmath, amssymb, graphicx, hyperref}
\usepackage{setspace}
\usepackage{cite}
\doublespacing

\title{A Theoretical Framework for Clustering Methods in Machine Learning}
\author{}
\date{}

\begin{document}

\maketitle

\begin{abstract}
Clustering is an essential unsupervised learning technique used for identifying natural groupings in data. This paper provides a theoretical framework for clustering methods, including K-Means, DBSCAN, and Agglomerative Hierarchical Clustering. The mathematical formulations, parameters, and performance evaluation metrics for these methods are discussed in detail, along with their comparative advantages and challenges.
\end{abstract}

\section{Introduction}
Clustering is a fundamental data analysis technique used in pattern recognition, image segmentation, anomaly detection, and various other applications. It partitions a dataset into clusters, ensuring that intra-cluster similarity is maximized while inter-cluster similarity is minimized. Various clustering algorithms exist, each employing distinct principles for data partitioning. This paper discusses three widely used clustering techniques: K-Means, DBSCAN, and Agglomerative Hierarchical Clustering.

\section{Clustering Methods}

\subsection{K-Means Clustering}
K-Means is a widely used centroid-based clustering algorithm that aims to minimize the variance within each cluster by iteratively updating cluster centroids. It is computationally efficient and effective for well-separated, spherical clusters \cite{Lloyd1982}.

\subsubsection{Mathematical Formulation}
Given a dataset with $n$ data points, the K-Means algorithm partitions them into $k$ clusters by minimizing the intra-cluster variance \cite{Kanungo2002}:
\begin{equation}
J = \sum_{i=1}^{k} \sum_{x \in C_i} ||x - \mu_i||^2,
\end{equation}
where $C_i$ represents cluster $i$, $\mu_i$ is the centroid of cluster $i$, and $||x - \mu_i||^2$ is the squared Euclidean distance.

\subsection{DBSCAN (Density-Based Spatial Clustering of Applications with Noise)}
DBSCAN is a density-based clustering algorithm that groups closely packed points in dense regions while marking outliers as noise. Unlike K-Means, DBSCAN does not require the number of clusters to be specified in advance and can detect clusters of arbitrary shape \cite{Ester1996}.

\subsubsection{Mathematical Formulation}
DBSCAN relies on two parameters: $\varepsilon$ (neighborhood radius) and $minPts$ (minimum points in a cluster). A point $p$ is a core point if:
\begin{equation}
|N_{\varepsilon}(p)| \geq minPts,
\end{equation}
where $N_{\varepsilon}(p)$ represents the set of points within distance $\varepsilon$ of $p$ \cite{Jain1988}.

\subsection{Agglomerative Hierarchical Clustering}
Agglomerative clustering is a bottom-up hierarchical clustering technique that iteratively merges clusters based on a linkage criterion \cite{Xu2005}.

\subsection{Performance Measures}
Several internal validation metrics are used to evaluate clustering effectiveness \cite{Jain1988}.

\section{Conclusion}
This paper has provided a theoretical overview of major clustering algorithms, their mathematical foundations, and performance evaluation techniques. Future work may involve empirical comparisons of these methods on real-world datasets.

\begin{thebibliography}{99}


\bibitem[17]{Lloyd1982}  
S. P. Lloyd, "Least squares quantization in PCM," \textit{IEEE Transactions on Information Theory}, vol. 28, no. 2, pp. 129–137, 1982.

\bibitem[18]{Kanungo2002}  
T. Kanungo, D. M. Mount, N. S. Netanyahu, C. D. Piatko, R. Silverman, and A. Y. Wu, "An efficient k-means clustering algorithm: Analysis and implementation," \textit{IEEE Transactions on Pattern Analysis and Machine Intelligence}, vol. 24, no. 7, pp. 881–892, 2002.

\bibitem[19]{Ester1996}  
M. Ester, H.-P. Kriegel, J. Sander, and X. Xu, "A density-based algorithm for discovering clusters in large spatial databases with noise," in \textit{Proceedings of the Second International Conference on Knowledge Discovery and Data Mining (KDD'96)}, Portland, OR, USA, 1996, pp. 226–231.

\bibitem[20]{Jain1988}  
A. K. Jain and R. C. Dubes, \textit{Algorithms for Clustering Data}. Prentice Hall, 1988.

\bibitem[21]{Xu2005}  
R. Xu and D. Wunsch, "Survey of clustering algorithms," \textit{IEEE Transactions on Neural Networks}, vol. 16, no. 3, pp. 645–678, 2005.

\end{thebibliography}



\end{document}
